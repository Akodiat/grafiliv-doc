
\chapter{Results}
Lorem ipsum dolor sit amet, consectetur adipisicing elit, sed do eiusmod tempor incididunt ut labore et dolore magna aliqua.

\section{System size impact on trophic levels}
%https://en.wikipedia.org/wiki/Trophic_level

%Snyggt att landskapet funkar och möjligen blir det lättare för detritusätarna att klara sig. Det slår mig att högre trofiska nivåer, som detritusätare, lider av en konversionsfaktor i energi: det måste finnas färre av dem. Finns det färre av dem är de mer utsatta för fluktuationer. Större system borde förbättra detta.

\tempText{
Energy conversion factor affecting detritus eaters? Try five different system sizes and investigate at what size detritus eaters can be sustained and if any organisms of even higher trophic levels emerge at larger system sizes.
}

\section{Influence of different habitats on emerging population}

\tempText{
Run simulations for equal time and parameter setup but one with terrain, one without, one in air, one in water, one in both
}

\section{Large timescale simulation}

\subsection{Species}
Since there is no sexual reproduction
% Det roligaste vore ju om det gick att följa en någorlunda lång körning där det uppstår några adapterade linjer - arter blir det ju inte eftersom de inte fortplantar sig sexuellt. Då skulle du kunna avbilda dem och beskriva vad de gör och hur deras fenotyper verkar vara adapterade. Sen skulle du kunna följa dem bakåt för att se hur tidigare former ser ut, när dessa anpassningar uppstod och så vidare. Sen skulle du kunna plocka ut en representant för varje sort och visa släktträdet. Hur såg den senaste gemensamma föregångaren till dem ut? Kanske displayat i ett träd. Hur har ekosystemet utvecklats? Har olika trofiska nivåer uppstått? Två nivåer har vi ju sett: sol och detritus - fattas då bara predatorer.

\tempText{
Run a simulation for 2 weeks or more to investigate if any other interesting features emerge after long-term evolution
}