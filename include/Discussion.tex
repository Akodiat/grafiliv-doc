\chapter{Discussion}

\tempText{
Add discussion of results... Some previous thoughts follow:
}

Two somewhat related parameters greatly influences the success of the system: environment and time. Without a habitable environment, that is having the parameters set to values where the organisms can survive and evolve, the population will, by circular definition, not survive.

Moreover, as long as the environment is hospitable enough for the organisms to survive, the emerging population will evolve to thrive in that environment. This is analogous to a planet being within the habitable zone of a star. But habitable to whom, one may ask? Other kinds of life would thrive in an environment we find hostile. Some extremofiles certainly already do [ref?] and it very much depends on what we are prepared to call life.
%evolving clay crystallites [4] nanodes, [18] Betau et al (open problems in alife)

As for time, even a very hostile environment might have life emerging and thriving, given enough time. In a simulation you could have random initial organisms are inserted while others die out. Even more importantly, evolution, at least in the biological sense, needed enormous amounts of time before the emergence of complex life.

\tempText{
replicators -> cells -> eukaryotes -> multicellularity -> organs -> invertebrates
}

Since millions of years were of simulation time was beyond the timeframe of this project, the model was constructed to start with multicellularity and to simulate the system at the cellular level. Still it took time before signs of adaption and the more interesting results appeared.

Because this delay of results (it often took more that a day to decide if a given run was viable), together with the difficult search in the huge space of environmental parameters, a long time had to be spent adjusting parameters.

\section{Stopping rule for parameter evaluation}
There is a dilemma of whether to abort a simulation or to leave it running and hope for better results. If something seem to be wrong with the simulation, is it more strategic to abort it and try another, possibly better environment setup? Or is it better to continue to run and instead hope that the extra time already spent running will be worth more for the end result?
%https://en.wikipedia.org/wiki/Secretary_problem

\section{Modelling level}
%Principen kan ju diskuteras - att det är viktigt att formulera modellen på rätt nivå. Är det för låg nivå kommer det aldrig igång, om det är för hög nivå så är design space för liten
An important thing to consider when modelling any system is which level to formulate the model on. A lower level model would increase the open-endedness of the system, allowing for a larger design space. However if it is too low there will not be enough time for interesting results to appear.

Originally, the model was run solely on a cell-level; with cells replicating individually. But such a setup proved to be too low-level for any interesting results to appear within a reasonable timeframe. The organisms were still intended to be multicellular, but that would require an emergent change of individuality, something the model did not allow for. Instead, the egg cells were introduced and the organism reproduction was centralised to the organism, not handled by each cell.

A model including mutable cell types, where 

\section{Future work} %Should be final section in discussion

\tempText{
Sexual reproduction (with crosover), even longer simulations, modelling on an even lower scale to create further open-endedness, etc.
}