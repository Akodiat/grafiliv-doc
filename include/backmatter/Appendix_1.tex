% CREATED BY DAVID FRISK, 2016
\chapter{Getting started with Grafiliv}
Grafiliv (short for \emph{grafikkortsliv}, graphics card life) is the program created for running the simulations. Source code and binaries can be found on GitHub: \url{https://github.com/Akodiat/grafiliv}.

In order to run a grafiliv simulation:

\begin{enumerate}
    \item Obtain binaries by either:
    \begin{enumerate}
    \item Compile the source code:
        \begin{enumerate}
            \item Download and install the Fluidix library at: \url{http://www.fluidix.ca/}
            \item Using the Fluidix app, compile \path{\fluidix\grafiliv\grafiliv.cu}
            \item If you want to compile the GrafilivViewer as well (instead of using the provided binary), download Unity3d \url{https://unity3d.com/} and open and compile the project at \path{\GrafilivViewer}
        \end{enumerate}
    \item  Download binaries from the Git repository:
        \begin{enumerate}
            \item Download and extract the zip-file corresponding to the version you want to run from \url{https://github.com/Akodiat/grafiliv/tree/master/app}
        \end{enumerate}
    \end{enumerate}
    \item Configure \path{\fluidix\grafiliv\config.txt} with the parameters you desire.
    \item If compiled for terrain, make sure \path{terrain.stl} is present in the same directory as \path{grafiliv.exe}
    \item Launch simulation by starting \path{grafiliv.exe}. If an earlier simulation was aborted, you have the option to restart it.
    \item Use \path{GrafilivViewer.exe}, also located in the same directory as \path{grafiliv.exe}, to inspect the simulated organisms
\end{enumerate}